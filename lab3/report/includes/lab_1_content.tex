\begin{center}
  \textbf{Отчёт лабораторной работы №\envReportLabNumber}
\end{center}

\textbf{Тема}:
<<\envReportTitle>>

\textbf{Цель}: приобрести навыки написания запросов выборки данных.

% = = = = = = = = = = = = = = = =

\begin{center}
  \textbf{Подготовка к лабораторной работе}
\end{center}

Перед тем как начать делать эту лабораторную работу, нужно спроектировать базу данных до 3-ей нормальной формы
и заполнить её данными, чтобы SELECT-ом получать данные.

Базу данных до 3-ей нормальной формы спроектировали во 2-ой лабораторной работе.
Я проектировал базу данных в SQL Power Architect. Результат на рисунке~\ref{fig:database}.

\begin{figure}[!h]
  \centering

  \includegraphics[width=16cm]
  {../sources/database/BD7.PO4.190333-02_05_00__logic_model.architect.pdf}

  \caption{Логическая модель}

  \label{fig:database}
\end{figure}

Я на Node JS написал код, который генерирует файл *.sql, которые содержит транзакции с INSERT-ами.
Например таблицу с осмотрами я заполнял за три кода.
Каждый рабочий день работаю несколько участков и врачи проводят прием либо в поликлинике, либо дома.

Я заполнил следующие таблицы рандомными данными:

\begin{itemize}
  \item справочник <<Диагнозы>>, результат на рисунке~\ref{fig:DE_CTL_Diagnosis};
  \item справочник <<Доктора>>, результат на рисунке~\ref{fig:DE_CTL_Doctors};
  \item справочник <<Гендеры>>, результат на рисунке~\ref{fig:DE_CTL_Genders};
  \item справочник <<Места осмотра>>, результат на рисунке~\ref{fig:DE_CTL_InspectionPlaces};
  \item справочник <<Лекарства>>, результат на рисунке~\ref{fig:DE_CTL_Medicines};
  \item справочник <<Пациенты>>, результат на рисунке~\ref{fig:DE_CTL_Patients};
  \item справочник <<Симптомы>>, результат на рисунке~\ref{fig:DE_CTL_Symptoms};
  \item оперативный документ <<Осмотр>>, результат на рисунке~\ref{fig:DE_DOC_Inspection};
  \item табличная часть <<Лекарства выписанные при осмотре>>, результат на рисунке~\ref{fig:DE_TAB_InspectionMedicines};
  \item табличная часть <<Симптомы выявленные при осмотре>>, результат на рисунке~\ref{fig:DE_TAB_InspectionSymptoms};
  \item табличная часть <<Симптомы выявленные при осмотре>>, результат на рисунке~\ref{fig:DE_TAB_MedicineSideEffects}.
\end{itemize}

\begin{figure}[p!h]
  \centering

  \includegraphics[height=5cm]
  {inc/DE_CTL_Diagnosis.png}

  \caption{Справочник <<Диагнозы>>}

  \label{fig:DE_CTL_Diagnosis}
\end{figure}

\begin{figure}[p!h]
  \centering

  \includegraphics[height=5cm]
  {inc/DE_CTL_Doctors.png}

  \caption{Справочник <<Доктора>>}

  \label{fig:DE_CTL_Doctors}
\end{figure}


\begin{figure}[p!h]
  \centering

  \includegraphics[height=5cm]
  {inc/DE_CTL_Genders.png}

  \caption{Справочник <<Гендеры>>}

  \label{fig:DE_CTL_Genders}
\end{figure}

\begin{figure}[p!h]
  \centering

  \includegraphics[height=5cm]
  {inc/DE_CTL_InspectionPlaces.png}

  \caption{Справочник <<Места осмотра>>}

  \label{fig:DE_CTL_InspectionPlaces}
\end{figure}

\begin{figure}[p!h]
  \centering

  \includegraphics[height=5cm]
  {inc/DE_CTL_Medicines.png}

  \caption{Справочник <<Лекарства>>}

  \label{fig:DE_CTL_Medicines}
\end{figure}

\begin{figure}[p!h]
  \centering

  \includegraphics[height=5cm]
  {inc/DE_CTL_Patients.png}

  \caption{Справочник <<Пациенты>>}

  \label{fig:DE_CTL_Patients}
\end{figure}

\begin{figure}[p!h]
  \centering

  \includegraphics[height=5cm]
  {inc/DE_CTL_Symptoms.png}

  \caption{Справочник <<Симптомы>>}

  \label{fig:DE_CTL_Symptoms}
\end{figure}

\begin{figure}[p!h]
  \centering

  \includegraphics[height=5cm]
  {inc/DE_DOC_Inspection.png}

  \caption{Оперативный документ <<Осмотры>>}

  \label{fig:DE_DOC_Inspection}
\end{figure}

\begin{figure}[p!h]
  \centering

  \includegraphics[height=5cm]
  {inc/DE_TAB_InspectionMedicines.png}

  \caption{Табличная часть <<Лекарства выписанные при осмотре>>}

  \label{fig:DE_TAB_InspectionMedicines}
\end{figure}

\begin{figure}[p!h]
  \centering

  \includegraphics[height=5cm]
  {inc/DE_TAB_InspectionSymptoms.png}

  \caption{Табличная часть <<Симптомы выявленные при осмотре>>}

  \label{fig:DE_TAB_InspectionSymptoms}
\end{figure}

\begin{figure}[p!h]
  \centering

  \includegraphics[height=5cm]
  {inc/DE_TAB_MedicineSideEffects.png}

  \caption{Табличная часть <<Симптомы выявленные при осмотре>>}

  \label{fig:DE_TAB_MedicineSideEffects}
\end{figure}

\newpage

\begin{center}
\textbf{Задание 1}
\end{center}

\lstinputlisting[language=sql]{../sql/select/1.sql}

% \lstinputlisting[language=sql]{../sql/select/2.sql}

\lstinputlisting[language=sql]{../sql/select/3.sql}

\lstinputlisting[language=sql]{../sql/select/4.sql}

\lstinputlisting[language=sql]{../sql/select/5.sql}

\lstinputlisting[language=sql]{../sql/select/6.sql}

\lstinputlisting[language=sql]{../sql/select/7.sql}

\lstinputlisting[language=sql]{../sql/select/8.sql}

\lstinputlisting[language=sql]{../sql/select/9.sql}

\lstinputlisting[language=sql]{../sql/select/10.sql}

\lstinputlisting[language=sql]{../sql/select/11.sql}

\lstinputlisting[language=sql]{../sql/select/12.sql}

\lstinputlisting[language=sql]{../sql/select/13.sql}

\lstinputlisting[language=sql]{../sql/select/14.sql}

\lstinputlisting[language=sql]{../sql/select/15.sql}

\newpage

\begin{figure}[p!h]
  \centering

  \includegraphics[height=5cm]
  {../sql/select/1-out.png}

  \caption{Выборка из задания 1}

  \label{fig:t1}
\end{figure}

\begin{figure}[p!h]
  \centering

  \includegraphics[height=5cm]
  {../sql/select/2-out.png}

  \caption{Выборка из задания 2}

  \label{fig:t2}
\end{figure}

\begin{figure}[p!h]
  \centering

  \includegraphics[height=5cm]
  {../sql/select/4-out.png}

  \caption{Выборка из задания 4}

  \label{fig:t4}
\end{figure}

\begin{figure}[p!h]
  \centering

  \includegraphics[height=5cm]
  {../sql/select/5-out.png}

  \caption{Выборка из задания 5}

  \label{fig:t5}
\end{figure}

\begin{figure}[p!h]
  \centering

  \includegraphics[height=5cm]
  {../sql/select/6-out.png}

  \caption{Выборка из задания 6}

  \label{fig:t6}
\end{figure}

\begin{figure}[p!h]
  \centering

  \includegraphics[height=5cm]
  {../sql/select/7-out.png}

  \caption{Выборка из задания 7}

  \label{fig:t7}
\end{figure}

\begin{figure}[p!h]
  \centering

  \includegraphics[height=5cm]
  {../sql/select/8-out.png}

  \caption{Выборка из задания 8}

  \label{fig:t8}
\end{figure}