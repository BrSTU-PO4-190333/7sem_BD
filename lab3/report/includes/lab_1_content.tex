\begin{center}
  \textbf{Отчёт лабораторной работы №\envReportLabNumber}
\end{center}

\textbf{Тема}:
<<\envReportTitle>>

\textbf{Цель}: приобрести навыки написания запросов выборки данных.

% = = = = = = = = = = = = = = = =

\begin{center}
  \textbf{Подготовка к лабораторной работе}
\end{center}

Перед тем как начать делать эту лабораторную работу, нужно спроектировать базу данных до 3-ей нормальной формы
и заполнить её данными, чтобы SELECT-ом получать данные.

Базу данных до 3-ей нормальной формы спроектировали во 2-ой лабораторной работе.
Я проектировал базу данных в SQL Power Architect. Результат на рисунке~\ref{fig:database}.

\begin{figure}[!h]
  \centering

  \includegraphics[width=16cm]
  {../sources/database/BD7.PO4.190333-02_05_00__logic_model.architect.pdf}

  \caption{Логическая модель}

  \label{fig:database}
\end{figure}

Я на Node JS написал код, который генерирует файл *.sql, которые содержит транзакции с INSERT-ами.
Например таблицу с осмотрами я заполнял за три кода.
Каждый рабочий день работаю несколько участков и врачи проводят прием либо в поликлинике, либо дома.

Я заполнил следующие таблицы рандомными данными:

\begin{itemize}
  \item справочник <<Диагнозы>>, результат на рисунке~\ref{fig:DE_CTL_Diagnosis};
  \item справочник <<Доктора>>, результат на рисунке~\ref{fig:DE_CTL_Doctors};
  \item справочник <<Гендеры>>, результат на рисунке~\ref{fig:DE_CTL_Genders};
  \item справочник <<Места осмотра>>, результат на рисунке~\ref{fig:DE_CTL_InspectionPlaces};
  \item справочник <<Лекарства>>, результат на рисунке~\ref{fig:DE_CTL_Medicines};
  \item справочник <<Пациенты>>, результат на рисунке~\ref{fig:DE_CTL_Patients};
  \item справочник <<Симптомы>>, результат на рисунке~\ref{fig:DE_CTL_Symptoms};
  \item оперативный документ <<Осмотр>>, результат на рисунке~\ref{fig:DE_DOC_Inspection};
  \item табличная часть <<Лекарства выписанные при осмотре>>, результат на рисунке~\ref{fig:DE_TAB_InspectionMedicines};
  \item табличная часть <<Симптомы выявленные при осмотре>>, результат на рисунке~\ref{fig:DE_TAB_InspectionSymptoms};
  \item табличная часть <<Симптомы выявленные при осмотре>>, результат на рисунке~\ref{fig:DE_TAB_MedicineSideEffects}.
\end{itemize}

\begin{figure}[p!h]
  \centering

  \includegraphics[height=5cm]
  {inc/DE_CTL_Diagnosis.png}

  \caption{Справочник <<Диагнозы>>}

  \label{fig:DE_CTL_Diagnosis}
\end{figure}

\begin{figure}[p!h]
  \centering

  \includegraphics[height=5cm]
  {inc/DE_CTL_Doctors.png}

  \caption{Справочник <<Доктора>>}

  \label{fig:DE_CTL_Doctors}
\end{figure}


\begin{figure}[p!h]
  \centering

  \includegraphics[height=5cm]
  {inc/DE_CTL_Genders.png}

  \caption{Справочник <<Гендеры>>}

  \label{fig:DE_CTL_Genders}
\end{figure}

\begin{figure}[p!h]
  \centering

  \includegraphics[height=5cm]
  {inc/DE_CTL_InspectionPlaces.png}

  \caption{Справочник <<Места осмотра>>}

  \label{fig:DE_CTL_InspectionPlaces}
\end{figure}

\begin{figure}[p!h]
  \centering

  \includegraphics[height=5cm]
  {inc/DE_CTL_Medicines.png}

  \caption{Справочник <<Лекарства>>}

  \label{fig:DE_CTL_Medicines}
\end{figure}

\begin{figure}[p!h]
  \centering

  \includegraphics[height=5cm]
  {inc/DE_CTL_Patients.png}

  \caption{Справочник <<Пациенты>>}

  \label{fig:DE_CTL_Patients}
\end{figure}

\begin{figure}[p!h]
  \centering

  \includegraphics[height=5cm]
  {inc/DE_CTL_Symptoms.png}

  \caption{Справочник <<Симптомы>>}

  \label{fig:DE_CTL_Symptoms}
\end{figure}

\begin{figure}[p!h]
  \centering

  \includegraphics[height=5cm]
  {inc/DE_DOC_Inspection.png}

  \caption{Оперативный документ <<Осмотры>>}

  \label{fig:DE_DOC_Inspection}
\end{figure}

\begin{figure}[p!h]
  \centering

  \includegraphics[height=5cm]
  {inc/DE_TAB_InspectionMedicines.png}

  \caption{Табличная часть <<Лекарства выписанные при осмотре>>}

  \label{fig:DE_TAB_InspectionMedicines}
\end{figure}

\begin{figure}[p!h]
  \centering

  \includegraphics[height=5cm]
  {inc/DE_TAB_InspectionSymptoms.png}

  \caption{Табличная часть <<Симптомы выявленные при осмотре>>}

  \label{fig:DE_TAB_InspectionSymptoms}
\end{figure}

\begin{figure}[p!h]
  \centering

  \includegraphics[height=5cm]
  {inc/DE_TAB_MedicineSideEffects.png}

  \caption{Табличная часть <<Симптомы выявленные при осмотре>>}

  \label{fig:DE_TAB_MedicineSideEffects}
\end{figure}

\newpage

\begin{center}
\textbf{Задание 1}
\end{center}

\textbf{Условие}:
Вывести данные о всех приемах (дату, продолжительность приема в минутах, место
осмотра, данные врача, данные пациента), которые были проведены между датами
01.01.19 и 20.02.19 (привести два варианта решения задачи).

\textbf{Что я сделал}:

Я связал таблицы Осмотры-МестаОсмотра, Осмотры-Доктора, Осмотры-Пациенты.

Я отобрал записи от даты 2019-01-01 00:00:00 по дату 2019-02-20 00:00:00.

Я вывожу поля
\begin{itemize}
  \item дата: стартовое время осмотра
  \item продолжительность в минутах: в таблице осмотр отнял конец осмотра от начала осмотра и функцией достал минуты
  \item место осмотра: взял и вывел поле наименование в таблице места осмотра
  \item данные\_врача: взял и сконкатенировал данные из таблицы докторов
  \item данные\_пациена: взял и сконкатенировал данные из таблицы пациентов
\end{itemize}

Я отсортировал по полю дата.

\lstinputlisting[language=sql]{../sql/task1/1.sql}

\textbf{Время выполнения}: 68-117 msec.

\textbf{Размер выборки}: 11174 rows.

\textbf{Результат}: часть выборки (в скриншоте LIMIT 24) изображена на рисунке~\ref{fig:t1}.

\begin{figure}[!h]
  \centering

  \includegraphics[height=5cm]
  {../sql/task1/1-out.png}

  \caption{Выборка из задания 1}

  \label{fig:t1}
\end{figure}

% = = = = = = = = = = = = = = = =

\begin{center}
  \textbf{Задание 2}
\end{center}
  
\textbf{Условие}:
Вывести названия всех лекарств, у которых в названии присутствует '3\%'.
  
\textbf{Что я сделал}:

Я взял таблицу лекарств.

Я вывел поле наименование.

Я отобрал записи, которые начинаются с букву З.

\lstinputlisting[language=sql]{../sql/task2/2.sql}

\textbf{Время выполнения}: 3-5 msec.

\textbf{Размер выборки}: 36 rows.

\textbf{Результат}: часть выборки (в скриншоте LIMIT 24) изображена на рисунке~\ref{fig:t2}.

\begin{figure}[!h]
  \centering

  \includegraphics[height=5cm]
  {../sql/task2/2-out.png}

  \caption{Выборка из задания 2}

  \label{fig:t2}
\end{figure}

% = = = = = = = = = = = = = = = =

\begin{center}
  \textbf{Задание 3}
\end{center}
  
\textbf{Условие}:
Вывести данные о врачах, обслуживших максимальное количество пациентов на дому.
  
\textbf{Что я сделал}:

Я связал таблицы Осмотры-МестаОсмотра, Осмотры-Доктора.

Я отобрал осмотры, которые проводились на дому.

Я вывел список врачей и их количество обслужанных пациентов,
затем из этого множества нашел максимальное количество.

\lstinputlisting[language=sql]{../sql/task3/3.sql}

\textbf{Время выполнения}: 71-104 msec.

\textbf{Размер выборки}: 1 row.

\textbf{Результат}: вся выборка изображена на рисунке~\ref{fig:t3}.

\begin{figure}[!h]
  \centering

  \includegraphics[width=16cm]
  {../sql/task3/3-out.png}

  \caption{Выборка из задания 3}

  \label{fig:t3}
\end{figure}

% = = = = = = = = = = = = = = = =

\begin{center}
  \textbf{Задание 4}
\end{center}
  
\textbf{Условие}:
Для каждого врача подсчитать общее время обслуживания пациентов в госпитале
  
\textbf{Что я сделал}:

Я связал таблицы Осмотр-Доктора, Осмотр-МестаОсмотра.

Я отобрал записи только по осмотру в поликлинике.

Я сгруппировал данные и вывел количество часов.

\lstinputlisting[language=sql]{../sql/task4/4.sql}

\textbf{Время выполнения}: 140-190 msec.

\textbf{Размер выборки}: 9 rows.

\textbf{Результат}: вся выборка изображена на рисунке~\ref{fig:t4}.

\begin{figure}[!h]
  \centering

  \includegraphics[width=16cm]
  {../sql/task4/4-out.png}

  \caption{Выборка из задания 4}

  \label{fig:t4}
\end{figure}

% = = = = = = = = = = = = = = = =

\begin{center}
  \textbf{Задание 5}
\end{center}
  
\textbf{Условие}:
Вывести диагнозы, которые не были поставлены ни одним врачом
  
\textbf{Что я сделал}:

В таблице осмотры у меня 325632 осмотра. В таблице диагнозы у меня 14629 диагнозов.
Из 325632 осмотра у меня за 3 года выявлено 14629 диагнозов, то есть все,
поэтому я возьму обределённый период, например, 1 января 2019 - 20 февраля 2019.

За два месяца (1 января 2019 - 20 февраля 2019) проведено 11734 осмотра и выявлено
7826 уникальных диагноза, тоесть не поставили диагнозов 6803 (14629-7826=6803).

Из таблицы осмотры я вывел диагнозы от 1 января 2019 до 20 февраля 2019.

Из таблицы диагнозы я отобрал те, которые были не поставлены от 1 января 2019 до 20 февраля 2019.

\lstinputlisting[language=sql]{../sql/task5/5.sql}

\textbf{Время выполнения}: 45-72 msec.

\textbf{Размер выборки}: 6803 rows.

\textbf{Результат}: часть выборки (в скриншоте LIMIT 24) изображена на рисунке~\ref{fig:t5}.

\begin{figure}[!h]
  \centering

  \includegraphics[width=16cm]
  {../sql/task5/5-out.png}

  \caption{Выборка из задания 5}

  \label{fig:t5}
\end{figure}

% = = = = = = = = = = = = = = = =

\begin{center}
  \textbf{Задание 6}
\end{center}
  
\textbf{Условие}:
В запросе для каждого врача подсчитать и вывести, начиная с даты 01.01.19, количество
пациентов каждого пола, а также количество пациентов, обслуженных не в госпитале
  
\textbf{Что я сделал}:

Объединил таблицы Осмотры-Пациенты, Пациенты-Гендеры, Осмотры-МестаОсмотра.

Отобрал записи начиная от 1 января 2019 года.

Сгруппировал записи по гендору пациента и месту осмотра.

\lstinputlisting[language=sql]{../sql/task6/6.sql}

\textbf{Время выполнения}: 97-118 msec.

\textbf{Размер выборки}: 4 rows.

\textbf{Результат}: вся выборка изображена на рисунке~\ref{fig:t6}.

\begin{figure}[!h]
  \centering

  \includegraphics[width=16cm]
  {../sql/task6/6-out.png}

  \caption{Выборка из задания 6}

  \label{fig:t6}
\end{figure}

\newpage

% = = = = = = = = = = = = = = = =

\begin{center}
  \textbf{Задание 7}
\end{center}
  
\textbf{Условие}:
Написать запрос, выводящий для каждого диагноза количество пациентов, название
самого диагноза, а также средний возраст пациентов диагноза

\textbf{Что я сделал}:

Я объединил таблицы Осмотры-Пациенты, Осмотры-Диагнозы.

Я сгруппировал выборку по ид дигноза.

Я подсчитал количество каждого дигноза отдельно.

Я подсчитал средний возвраст для каждого диагноза.

\lstinputlisting[language=sql]{../sql/task7/7.sql}

\textbf{Время выполнения}: 194-235 msec.

\textbf{Размер выборки}: 14629 rows.

\textbf{Результат}: вся выборка изображена на рисунке~\ref{fig:t7}.

\begin{figure}[!h]
  \centering

  \includegraphics[width=16cm]
  {../sql/task7/7-out.png}

  \caption{Выборка из задания 7}

  \label{fig:t7}
\end{figure}

\newpage

% = = = = = = = = = = = = = = = =

\begin{center}
  \textbf{Задание 8}
\end{center}
  
\textbf{Условие}:
Вывести данные о врачах, у которых существует хотя бы один пациент старше 100 лет.

\textbf{Что я сделал}:

\lstinputlisting[language=sql]{../sql/select/8.sql}

\lstinputlisting[language=sql]{../sql/select/9.sql}

\lstinputlisting[language=sql]{../sql/select/10.sql}

\lstinputlisting[language=sql]{../sql/select/11.sql}

\lstinputlisting[language=sql]{../sql/select/12.sql}

\lstinputlisting[language=sql]{../sql/select/13.sql}

\lstinputlisting[language=sql]{../sql/select/14.sql}

\lstinputlisting[language=sql]{../sql/select/15.sql}

\newpage
